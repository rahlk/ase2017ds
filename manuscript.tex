\documentclass[conference]{IEEEtran}
\usepackage{blindtext}
\usepackage{graphicx}
\usepackage{amsmath}
\usepackage{array}
\usepackage{mdwmath}
\usepackage{mdwtab}
\usepackage{eqparbox}
\usepackage[tight,footnotesize]{subfigure}
\usepackage[caption=false, font=footnotesize]{subfig}
\usepackage{stfloats}
\usepackage{url}


\begin{document}
\title{Actionable Analytics in Software Engineering\\[-0.2cm]}
\author{Rahul Krishna\\
\IEEEauthorblockA{Department of Computer Science\\
North Carolina State University\\
Email: rkrish11@ncsu.edu}}

% make the title area
\maketitle


\begin{abstract}
% Brief premise
% A common solution
% A challenge with this solution
% An alternative option
% The purpose of this thesis
% Validation plan

The primary motivation of much of software analytics is decision making. How do you make these decisions? Should one make decisions based on lessons that arise from within a particular project or should one generate these decisions from across multiple projects? If sufficient data is not available within a project, can practitioners learn lessons from other projects? This work is an attempt to answer these questions. Our work was motivated by a realization that much of the current generation software analytics tools focus primarily on prediction algorithms. Indeed prediction is a useful task, but it is usually followed by ``planning'' about what actions need to be taken. This research seeks to address the planning task by seeking methods that support actionable analytics that offer clear guidance on \textit{what to do} within the context of a specific software project. Specifically, we propose the XTREE algorithm for generating a set of actionable plans. Each of these plans, if followed will improve the quality of the software project.

\end{abstract}

\begin{IEEEkeywords}

\end{IEEEkeywords}


\section{Introduction}

Over the past decade, advances in AI have enabled a widespread use of data analytics in software engineering. For example, we can now estimate	how long it would take to integrate the new code~\cite{czer11},	where bugs are most likely to occur~\cite{ostrand04,Menzies2007a}, or how much effort it will take to develop a software package~\cite{turhan11,koc11b}, etc.

\section{Conclusion}

\section*{Acknowledgment}

\bibliographystyle{IEEEtran}
\bibliography{ref}

\end{document}


