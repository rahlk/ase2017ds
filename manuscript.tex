\documentclass[conference]{IEEEtran}
\usepackage{blindtext}
\usepackage{graphicx}
\usepackage{amsmath}
\usepackage{array}
\usepackage{mdwmath}
\usepackage{mdwtab}
\usepackage{eqparbox}
\usepackage{stfloats}
\usepackage[]{hyperref}
% Shortcuts
\newcommand{\bi}{\begin{itemize}}
\newcommand{\ei}{\end{itemize}}
\newcommand{\be}{\begin{enumerate}}
\newcommand{\ee}{\end{enumerate}}
\newcommand{\tion}[1]{\S\ref{sect:#1}}
\newcommand{\fig}[1]{Figure~\ref{fig:#1}}
\newcommand{\tab}[1]{Table ~\ref{tab:#1}}
\newcommand{\eq}[1]{Equation~\ref{eq:#1}}

\begin{document}
\title{Learning Effective Changes for  Software Projects\\[-0.8cm]}
\author{Rahul Krishna\\
\IEEEauthorblockA{Department of Computer Science\\
North Carolina State University\\
Email: rkrish11@ncsu.edu}}

% make the title area
\maketitle


\begin{abstract}
% Brief premise
% A common solution
% A challenge with this solution
% An alternative option
% The purpose of this thesis
% Validation plan

The primary motivation of much of software analytics is decision making. How do you make these decisions? Should one make decisions based on lessons that arise from within a particular project or should one generate these decisions from across multiple projects? If sufficient data is not available within a project, can practitioners learn lessons from other projects? This work is an attempt to answer these questions. Our work was motivated by a realization that much of the current generation software analytics tools focus primarily on prediction algorithms. Indeed prediction is a useful task, but it is usually followed by ``planning'' about what actions need to be taken. This research seeks to address the planning task by seeking methods that support actionable analytics that offer clear guidance on \textit{what to do} within the context of a specific software project. Specifically, we propose the XTREE algorithm for generating a set of actionable plans. Each of these plans, if followed will improve the quality of the software project.

\end{abstract}

\begin{IEEEkeywords}
Data mining, actionable analytics, bellwethers, defect prediction.
\end{IEEEkeywords}


\section{Introduction}

Over the past decade, advances in AI have enabled a widespread use of data analytics in software engineering. For example, we can now estimate	how long it would take to integrate the new code~\cite{czer11}, where bugs are most likely to occur~\cite{ostrand04,Menzies2007a}, or how much effort it will take to develop a software package~\cite{turhan11,koc11b}, etc. Despite these successes, there are some operational shortcomings with certain software analytic tools.  Business users lament that most software analytics tools, ``Tell us what \textit{is}. But they don't tell us \textit{what to do}''. A similar concern was raised by several researchers at a recent workshop on ``Actionable Analytics'' at 2015 IEEE conference on Automated Software Engineering~\cite{hihn15}. 

For example, most software analytics tools in the area of detecting software defects are mostly \textit{prediction} algorithms such as support vector machines~\cite{cortes95}, naive bayes classifiers~\cite{lessmann08}, logistic regression~\cite{lessmann08}, decision trees~\cite{dtrees}, etc. These prediction algorithms report what combinations of software project features predict for say the number of defects. But this is different task to \textit{planning}, which answers a more pressing question: what to {\em change} in order to {\em improve} quality. Accordingly, in this research, we seek tools that offer clear guidance on what to do in a specific project.

The tool assessed in this paper is the XTREE \textit{planning} tool~\cite{krishna17a}. XTREE employs a $cluster+contrast$ approach to planning where it (a) \textit{Clusters} different parts of the software project based on a quality measure (e.g. the number of defects); (b) Reports the \textit{contrast sets} between neighboring clusters. Each of these contrast sets represent the difference between these clusters and they can  be interpreted as plans, i.e., 
\begin{itemize}
	    \item If your current project falls into cluster $C_1$,
	    \item Some neighboring cluster $C_2$ has better quality.
	    \item Then the difference {\em $\Delta=$ $C_2$ - $C_1$} is a {\em plan} for changing a  project such that it \textit{might} have   higher quality.
\end{itemize}

% In several applications, local data is scarce. 

\section{Conclusion}

\section*{Acknowledgment}

\bibliographystyle{IEEEtran}
<<<<<<< HEAD
\bibliography{manuscript}

\end{document}
=======
\bibliography{references}

\end{document}
>>>>>>> 29f10521be6ca16ebbc481b37ddb7be90684a1bf
